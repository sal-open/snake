\documentclass[a4paper,12pt]{report}
\usepackage[utf8]{inputenc}
\usepackage{listings}
\usepackage{graphicx}
\usepackage{color}
\usepackage{forest}
\begin{document}

\begin{titlepage}
    \centering
    {\Huge Relazione progetto Sn@ke\par}
    \vspace{1cm}
    {\large Corso di Programmazione\par}
    \vspace{1cm}
    {\large Pietro Tombaccini, 0001172381\par}
    {\large Salvatore Bruzzese, 0001161573\par}
    {\large Andrei Pirva Cosmin, Matricola\par}
    \vfill
    {\large Anno Accademico 2024/2025\par}
\end{titlepage}

\tableofcontents
\chapter{Introduzione e analisi del problema}
\section{Introduzione}
\section{Analisi del problema}
\section{Divisione compiti}

\chapter{Progettazione}
\section{Architettura generale del progetto}
\begin{forest}
	for tree={
		edge={->},
		align=center
	}
	[Menu
		[Gioco
			[Serpente]
			[Mela]
			[Pausa]
			[Livello
				[Velocità]
				[Tempo]
			]
		]
		[Classifica]
		[Informazioni]
		[Esci]
	]
\end{forest}
\section{La classe Snake}
\begin{lstlisting}[language=C++, caption={Metodo move della classe Snake}]
void Snake::move(WINDOW *win, int ch, int mult) {
    // ...
}
\end{lstlisting}
\section{La classe Game}
\chapter{Implementazione}
\chapter{Test e risultati}
\chapter{Conclusioni}
\appendix
\chapter{Manuale d'uso}

\end{document}